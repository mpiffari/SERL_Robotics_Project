Andiamo qui a riassumere alcuni passi operativi che sono stati realizzati durante lo sviluppo ed il proseguimento del progetto.

\section{Installazione driver camera}
Per scaricare i driver della camera è necessario registrarsi al sito https://en.ids-imaging.com, una volta scaricati i driver si procede all'installazione eseguendo il seguente:
\begin{lstlisting}
	sudo sh ./ueyesdk-setup*.run
\end{lstlisting}
Dopo l'installazione, a meno di un riavvio, è necessario avviare il demone della camera:
\begin{lstlisting}
	sudo /etc/init.d/ueyeusbdrc start 
\end{lstlisting}
Come specificato nel capitolo ~\ref{section:deamonCameraUeye}, il demone ueyeusbdrc deve essere sempre attivo, potrebbe non esserlo nel caso di un plug-in a caldo (nel caso la camera fosse connessa tramite porta ethernet si deve usare il demone ueyeehtdrc).





\section{Avvio del codice}
Nella nostra esperienza ci è stato molto comodo andare ad utilizzare non il classico terminale messo a disposizione da Ubuntu, ma bensì ci siamo appoggiati all'utilizzo di \textit{terminator}, il quale permette una migliore gestione di terminali multipli.

Ecco una sequenza di comandi da inserire da terminale per poter andare ad eseguire il codice. 
\begin{lstlisting}

Scrivere su tutti terminali (e su tutti quelli che verranno aperti) i seguenti comandi per il \textit{bash} del progetto:
	- CMD_1: cd ArrowFinder/devel/; . setup.bash
	- CMD_2: cd ..

Terminale 1:
	- Caricare il bash da devel (CMD_1 / CMD_2)
	- CMD_3: roscore

Terminale 2:
	- Caricare il bash da devel (CMD_1 / CMD_2)
	- CMD_4: rosrun arrow_finder arrow_finder_node
	
Terminale 3:
	- Caricare il bash da devel (CMD_1 / CMD_2)
	- CMD_5: cd ~/ArrowFinder/src/ueye_cam/launch
	- CMD_6: roslaunch debug.launch 

Terminale 4:
	- CMD_7: rosrun topic_tools throttle messages /camera/image_raw 5.0 
\end{lstlisting}
Da evidenziare come, il comando eseguito sul terminale 4, risulta essere utile nel caso in cui sia necessario andare a temporizzare, in maniera automatica e gestita completamente da ROS, la lettura dell'immagine